%% The following is a directive for TeXShop to indicate the main file
%%!TEX root = ../diss.tex

\chapter{HPG}
\label{ch:HPG}

\section{Preface}

Attribution: Material published elsewhere (or in press) must be identified and suitably acknowledged in both the text and the Preface.
Collaborative publications, in which the student is one of several authors/creators, are permitted.
In every case, the Preface must clearly describe the student's contribution to the research and creation—including, where applicable, the student's role in publications with several authors.

\textbf{Multi-modal magnetic resonance imaging and histology of vascular function in xenografts using macromolecular contrast agent hyperbranched polyglycerol (HPG-GdF)}

\section{Abstract}

Macromolecular gadolinium (Gd)-based contrast agents are in development as blood pool markers for MRI.
HPG-GdF is a 583 kDa hyperbranched polyglycerol doubly tagged with Gd and Alexa 647 nm dye, making it both MR and histologically visible.
In this study we examined the location of HPG-GdF in whole-tumor xenograft sections matched to in vivo DCE-MR images of both HPG-GdF and Gadovist.
Despite its large size, we have shown that HPG-GdF extravasates from some tumor vessels and accumulates over time, but does not distribute beyond a few cell diameters from vessels.
Fractional plasma volume (fPV) and apparent permeability–surface area product (aPS) parameters were derived from the MR concentration–time curves of HPG-GdF.
Non-viable necrotic tumor tissue was excluded from the analysis by applying a novel bolus arrival time (BAT) algorithm to all voxels.
aPS derived from HPG-GdF was the only MR parameter to identify a difference in vascular function between HCT116 and HT29 colorectal tumors.
This study is the first to relate low and high molecular weight contrast agents with matched whole-tumor histological sections.
These detailed comparisons identified tumor regions that appear distinct from each other using the HPG-GdF biomarkers related to perfusion and vessel leakiness, while Gadovist-imaged parameter measures in the same regions were unable to detect variation in vascular function.
We have established HPG-GdF as a biocompatible multi-modal high molecular weight contrast agent with application for examining vascular function in both MR and histological modalities.

\section{Introduction}

The vascular network in tumor tissue is abnormal, often resulting in vessels that have variable flow rates and high permeability relative to blood vessels in normal tissue~\cite{McDonald:2002ut}.
Dynamic contrast enhanced magnetic resonance imaging (DCE-MRI) is a useful tool for non-invasively assessing tumor vasculature by imaging and measuring concentrations of contrast agent (CA) delivered to tumors by the vessels (2–4).
The appeal of a dynamic, non-invasive approach for measuring tumor vascular function in the clinic is clear.
Such data is applicable in the field of assessing treatment response for vascular targeting therapies~\cite{OConnor:2012ie,Barrett:2006jx,Leach:2003fy}.
In addition to utility as a treatment biomarker, vascular function data may be able to predict which tumors are likely to respond to therapy~\cite{DeBruyne:2012cq,Kelly:2011cf,Bains:2009hh}, or which regions or tumors have greater heterogeneity in their microenvironment~\cite{OConnor:2011jm,Alic:2011hw}.
The issue of limited access for anticancer drugs in solid tumors is significant~\cite{Minchinton:2006gs}, particularly given the efforts to create nanoparticle therapeutics that target the tumor via the enhanced permeability and retention (EPR) effect~\cite{Jain:2001uf,Chauhan:2011fi}.
A translatable imaging protocol and suitable contrast agent that yields meaningful, reproducible biomarkers of vascular function could be widely useful in these areas of cancer research.

Low molecular weight (MW) Gd(III)-based, chelated CAs such as Gadovist (MW = 605 Da) exhibit short half-lives and rapid renal clearance~\cite{Weinmann:1984gv}.
With the exception of brain tissue containing a functional blood–brain barrier, low MW Gd agents currently in clinical use diffuse across the vascular endothelium in most normal and neoplastic tissues.
Therefore, a well-established shortcoming of low MW contrast agents is the difficulty of attributing local signal enhancement specifically to either vascular perfusion or vessel permeability.
The ability to characterize physiologically relevant biomarkers of vascular function is desirable for studying the effects of anti-angiogenic treatments in tumors (described in a comprehensive review regarding DCE-MRI and anti-vascular therapies in cancer~\cite{OConnor:2012ie}).
Development and application of macromolecular contrast agents (MCAs) attempt to improve upon DCE-MRI assessment of vascular function by relying on the primarily intravascular nature of MCAs.
High MW contrast agents and therapeutics unable to diffuse across the endothelium selectively extravasate from large pores and inter-endothelial cell gaps that characterize the abnormal vessels in the tumor microenvironment~\cite{McDonald:2002ut,Hashizume:2000bq}.
MCAs commonly used in cancer research include albumin, dextran polymers, and PAMAM or PPI dendrimers conjugated with DTPA/DOTA-Gd3+ chelates as described by Tang et al.~\cite{Tang:2013fi}.
The size of MCAs in use or under development ranges considerably, but may be as small as a 90 kDa albumin conjugate, or one of a range of dextran sizes~\cite{Barrett:2006jx}.
Particles less than 5 nm in size have been found to leak rapidly from tumor vasculature, whereas those in the 5–8nm range are limited to leaking from hyperpermeable vessels; those greater than 8nm are thought to have minimal leakage~\cite{Kobayashi:2004vq,Sato:2001tt}.
The most commonly used MCAs in preclinical research are albumin based; however, these are not translatable to the clinic due to immunogenicity concerns~\cite{Ogan:1987tg}.
Signal enhancement is linked to contrast agent concentration within the tumor blood vessels at early time points after injection for MCAs that remain intravascular, such that tumor plasma volumes (Vp) can be determined.
Rates of MCA leakage from hyperpermeable vessels may then be modeled to evaluate permeability, since extravascular accumulation of the agents will manifest as increased enhancement in repeat images~\cite{Ogan:1987tg,Turetschek:2004bw}.
This analysis is dependent on the assumption that the MCAs have unidirectional flow and do not leak back into the plasma, and that the concentration in the plasma is constant, creating a permeabilitylimited environment.
In this study we investigated a multi-modal, high MW contrast agent, HPG-GdF: hyperbranched polyglycerol (HPG) molecules doubly labeled with Gd-DOTA and a fluorescent marker.
HPGs are soluble, globular, asymmetrical, have low immunogenicity and are highly biocompatible molecules with low polydispersity~\cite{Saatchi:2012hc,Kainthan:2006ce,Saatchi:2012gc}.
HPG has been previously tested as a human serum substitute~\cite{Kainthan:2008ek} and as a drug delivery vehicle due to its versatility as a chemical, such that drugs, Gd chelates, fluorescent and radiolabels may all be attached to it~\cite{Shenoi:2013id}.
Many other MCAs, including Gd-albumin, are highly viscous, which can limit the applicable dose~\cite{Imranulhaq:2012ij}.
The MW of HPG-GdF is tunable to required applications, and a biodegradable version of HPG is also available for potential use~\cite{Shenoi:2013id}.
The HPG-GdF described in this study is 583kDa and 8–10nm in diameter; synthesis of HPG-GdF has previously been described~\cite{Saatchi:2012hc}.
A significant advantage of HPG-GdF is that it is a multi-modal agent, with both fluorescent and Gd-chelate labels that permit histological validation of observations made using MRI, including determining the degree to which the agent extravasates from the vasculature.
Previous studies have also used 111In as a SPECT label with utility for biodistribution studies~\cite{Saatchi:2012hc}.
In this work, we employed comprehensive histological methods to investigate the microregional location of HPG-GdF in two human colorectal xenograft models, and used this information to interrogate observations made non-invasively using DCE-MR imaging of the same agent in the same tumors.

\section{Methods}

\subsection{Mice and tumors}

Female NOD/SCID mice were bred and housed in institutional animal facilities; experiments in this study were approved by the Animal Care Committee of the University of British Columbia.
Fiducial markers were constructed of PE-50 polyethylene tubing (inner diameter, 0.58mm) and filled with paraffin wax and saline, creating an MR-visible interface~\cite{Bains:2009hh}.
Marker tubes were implanted subcutaneously in the sacral region of mice, in a craniocaudal orientation, 2 days prior to subcutaneous implantation of tumors.
Both histological and MR modalities imaged slices in the plane perpendicular to the marker tube to minimize angular differences between serial MR image slices obtained over multiple sessions, and for corresponding cryosections in histological processing.
HCT116 or HT29 human colorectal carcinoma cells obtained from the American Type Culture Collection (ATCC) were implanted near the fiducial tubes such that the tumors grew around the tubes.
Tumors were used when diameters reached 8–12 mm.
Mice were anaesthetized with isoflurane for the duration of imaging sessions.
Animals were positioned supine on the custom surface coil apparatus fitted with a lid lined by a temperature-controlled, water-filled heating blanket.
Body temperature and respiration rate were monitored throughout imaging.
Following their final scan animals were administered a 35 $\mu$L intravenous dose of 0.6 mg/mL carbocyanine (DioC7(3); Molecular Probes, Eugene, OR, USA) in 75\% dimethylsulfoxide as a fluorescent dye indicator of vessel perfusion in histological measures, and euthanized 5min after injection.
Some animals received HPG-GdF and tumors were collected at early time points for histological analysis only, with no MR-imaging.
Tumors were embedded and frozen vertically in optimum cutting temperature medium (OCT; Tissue-TEK) using their fiducial markers for guidance.

\subsection{Contrast agents and dosage}

Hyperbranched polyglycerol (HPG-GdF) (Fig.~\ref{hpgpaper:fig1}): 583 kDa HPG was synthesized at the University of British Columbia (D. Brook’s laboratory, Department of Chemistry) with a narrow polydispersity of PDI = 1.01 by ring-opening multibranching polymerization of glycidol using dioxane as the reaction medium, according to a published procedure~\cite{Kainthan:2006ce}.
HPG was derivatized with p-NH2-benzyl-DOTA (Macrocyclics, Dallas, TX, USA) at 20$\mu$g Gd per mg HPG and tagged with Alexa Fluor 647 (Invitrogen Life Technologies, Burlington, ON, Canada) as previously described (20).
The biological half-life of HPG-Gd (with no fluorescent tag) in mice has previously been examined in biodistribution studies and was reported as 32.6 h (20).
HPG-GdF was administered as a 6 $\mu$L/g bolus dose from 100 mg/mL (0.2 mM) using an intravenous (i.v.) catheter; therefore, the administered molar dose of HPG was 1.2 nmol/g and that of the chelated Gd(III) was 240–360 nmol/g (determined according to an estimated 200–300 chelates per HPG molecule (20)).
Assuming a blood volume of 5\% of mouse body weight, peak blood concentration of HPG-GdF is 24$\mu$M.
This value has been used to normalize relative tissue concentrations (Fig.~\ref{hpgpaper:fig2}(B)) and to calculate the fractional plasma volume (fPV; see Section 2.4).
The relaxivity of HPG-GdF has previously been measured and reported to be 1075 mM$^{-1}$s$^{-1}$1, which is approximately 300 times greater than that of Gadovist (3.58 mM$^{-1}$1 s$^{-1}$1) (20).
Gadovist (Bayer Healthcare, Toronto, ON, Canada; 607.4 Da) was administered by i.v. catheter as a 5 $\mu$L/g bolus dose from 60 mM solution, for an administered molar dose of Gadovist of 300 nmol/g.
All in-scanner i.v.
injections were performed using a power injector at a rate of 1 mL/min.
An extended fluid line connected to the i.v.
catheter secured to the tail vein of the mice permitted remote initiation of injections outside the scanner room.
Injections included a small volume (<40 $\mu$L) of heparinized saline followed by the contrast agent, which was then followed by a 20 $\mu$L flush of heparinized saline.

\begin{figure}[htbp]   
 \begin{center}  
 \includegraphics[width=\textwidth]{hpg/hpg-images/hpg_fig1-hpgdf.pdf}
 \caption{HPG-GdF. The 583 kDa globular Hyperbranched PolyGlycerol (HPG) molecules are derivatized with p-NH$_2$-benzyl-DOTA (Macrocyclics) at 20 $\mu$g Gd per mg HPG (approximately 300 chelates per molecule) and tagged with Alexa Fluor 647 dye, as previously described (\cite{Saatchi:2012hc}.}  
 \label{hpgpaper:fig1}  
 \end{center}
\end{figure}

\subsection{MRI acquisition}

All MRI experiments were performed at the UBC MRI Research Centre on a 7T Bruker BioSpec 70/30 scanner at room temperature with a combination of volume (transmit)/surface (receive) coil.
Each imaging session began with axial RARE T$_2$-weighted images for morphological reference and precise alignment of imaging plane.
Further T$_1$-weighted RARE images were acquired for qualitative assessment of contrast agent distribution.
During the first scanning session, T$_1$ and flip angle maps were acquired prior to a DCE-MRI experiment, followed by another T$_1$ measurement.
At each imaging session slice location and orientation were adjusted to match previous sessions.
T$_1$ measurements and flip angle mapping were performed using a multi-slice FLASH variable flip angle experiment (FLASH TR/TE = 500/2.75, FA = 10$^{\circ}$, 20$^{\circ}$, 30$^{\circ}$, 40$^{\circ}$, 50$^{\circ}$, 60$^{\circ}$, 70$^{\circ}$, 80$^{\circ}$, 90$^{\circ}$, 100$^{\circ}$, 110$^{\circ}$, 120$^{\circ}$, 130$^{\circ}$, 140$^{\circ}$, 150$^{\circ}$, 160$^{\circ}$, 170$^{\circ}$, 180$^{\circ}$, 190$^{\circ}$, 200$^{\circ}$, 215$^{\circ}$) and data were fit simultaneously for T$_1$ and the B1 scaling factor map.
DCE-MRI data was collected at 2.24 s time resolution (FLASH; TR/TE = 35/2.75; FA = 40; NR = 1200).
T$_1$ and DCE-MRI experiments all had identical geometry (matrix = 128 × 64; three slices; voxel size = 0.33 mm × 0.297 mm × 1.5 mm; 2.5 mm slice separation).
A follow-up T$_1$ measurement was performed (FLASH; TR/TE = 35/2.75; FA = 10$^{\circ}$, 20$^{\circ}$, 30$^{\circ}$, 40$^{\circ}$, 50$^{\circ}$, 60$^{\circ}$, 80$^{\circ}$, 100$^{\circ}$, 120$^{\circ}$) and the B$_1$ scaling factor map from the baseline acquisition was used to determine post-contrast T$_1$ (assuming that B$_1$ scaling does not change due to contrast injection).
Difference maps of relaxation rates $\delta$R$_1$ = 1/T$_1$(post-contrast) – 1/T$_1$(precontrast) were constructed for a measure of contrast agent concentration.
Animals received two DCE-MRI scans 24–48h apart, with Gadovist administered and scanned at the first session and HPG-GdF administered and scanned at the subsequent session with tumors collected for histological processing at about 60 min post administration of their HPG-GdF.

\begin{figure}[htbp]   
 \begin{center}  
 \includegraphics[width=0.7\textwidth]{hpg/hpg-images/hpg_fig2-bat.png}
 \caption{. MR-derived parameters to measure vascular function using HPG-GdF: bolus arrival time (BAT), fractional plasma volume (fPV) and apparent permeability–surface area product (aPS). (A) Sample curve showing change in concentration of CA as a function of scan time. The first violation (circle) is identified by Rule (1) at Frame 31, corresponding to a BAT of t = 77.5 s. The centre (mean) and upper (UCL) and lower (LCL) control limits ($\pm$3 SDs) are drawn for reference. Sample BAT parameter maps from HPG-GdF and Gadovist obtained from the same HT29 xenograft imaged 24h apart show that HPG-GdF is slower to arrive; both CAs arrive most quickly at the tumor margins. (B) HPG-GdF enhancement curve shown for a full imaging period where the bolus arrival is seen as a significant jump in enhancement from which the fPV is derived as the concentration at the start relative to the plasma concentration, since the CA is largely intravascular at that early time. The aPS is the slope of the enhancement after the bolus arrival. The concentration curves are shown as ratio of tissue concentration to blood peak concentration (24 $\mu$M) (C) The fraction of MR-measured HPG-GdF BAT-enhancing voxels has a negative association with the proportion of necrotic tissue determined in histological sections.}  
 \label{hpgpaper:fig2}  
 \end{center}
\end{figure}

\subsection{MRI data analysis}

Regions of interest (ROIs) were drawn on T2-weighted RARE images to outline the tumor using ImageJ (NIH), and all other MR analysis was performed using MATLAB (MathWorks, R2009a) and Python.
T$_1$ and flip-angle maps were calculated from variable flip-angle data with a slice-profile correction based on simulations described by Parker et al.~\cite{Parker:2001wj}.
The same method was extended to provide time-dependent T$_1$ and concentration–time in DCE data series.
Areas under the curves (AUCs) were numerically integrated starting from the bolus arrival time, or, for tumor-averaged AUC, starting from the common injection time point and extended to the indicated time points (1 and 37 min).
Bolus arrival time (BAT) is the time when detectable signal enhancement begins for a voxel (Fig.~\ref{hpgpaper:fig2}(A)) due to contrast agent arrival.
Based on the control-chart decision criterion and the Western Electric decision rules from MATLAB’s statistical toolbox~\cite{Shewhart:1931tq}, voxel enhancement was detected as a positive change from baseline signal for three consecutive timepoints (frames) (i) in the same direction, (ii) starting 5 s before the time of injection or later and (iii) for at least 10\% of all timepoints following initial change of signal away from baseline.
Therefore, voxels with a finite BAT and for which at least 10\% of the following intensities were classified as enhancing by the BAT criteria were called enhancing voxels, whereas all other voxels were labeled as non-enhancing.

A change from baseline was determined to have occurred
when any one of the following inclusion criteria were met:

\begin{enumerate}
	\item any point fell outside of 3 SDs from the average baseline concentration, or
	\item two of three consecutive points fell outside of the 2 SD limit on the same side of the mean, or
	\item four of five consecutive points fell outside of the 1 SD line, on the same side of the centre line, or
	\item eight consecutive points all fell on one side of the centre line.
\end{enumerate}

In the illustrated example (Fig.~\ref{hpgpaper:fig2}(A)), a change from the mean was identified at Frame 31 and the two following scans by Rule (1); therefore, the BAT point was selected as Frame 30.
The centre line (mean of baseline) and upper and lower control limits ($\pm$3 SDs) are drawn for reference.

\subsubsection{Pharmacokinetic modeling of DCE-MRI data}

Gadovist.
The extended Tofts model was used and three parameters resulted: K${trans}$, vE, and vP~\cite{Sourbron:2011ce}.
The arterial input function (AIF) for this model was determined previously by our laboratory using a projection-based method~\cite{Moroz:2013ee}.
HPG-GdF.
A two-parameter linear model was applied~\cite{Pathak:2005gu}.
Two parameters were used to characterize the MCA time curves: (1) the rapid increase at the time of injection (related to the fPV) and (2) the slope of the later enhancement (aPS) (Fig.~\ref{hpgpaper:fig2}(B)).
The relative plasma volume could be determined from the ratio of concentrations in the voxel of interest to the concentration in whole blood in the seconds after injection, since extra-vascular spread of the agent was negligible at this time.
The slope of the concentration time curve following contrast arrival is proportional to the permeability–surface area product (PS) when the assumption of a permeability-limited environment is valid.
However, in the highly variable tumor microenvironment, extravasation of contrast agent may deplete the intra-vascular concentration appreciably in conditions of high permeability, which would increase the relative contribution of perfusion to the composite measure of PS.
To stress that the interpretation of this slope value depends on the assumption of a permeability-limited environment, we term the slope the apparent permeability–surface area product (aPS)~\cite{DaldrupLink:2004gy,Dafni:2002kb}.

\subsection{Histology}

Cryosections of 10 $\mu$m were obtained along the plane perpendicular to the fiducial marker at depths corresponding to MRimaged slices.
Sections were imaged for DiOC7(3) and HPGGdF native fluorescence and fixed in acetone-methanol for 10min prior to staining and re-imaging for CD31 (PECAM/ CD31) and Hoechst 3342, labeling vascular endothelium and cell nuclei, respectively.
Sections were imaged as previously described~\cite{Kyle:2007ch} using a system of tiling adjacent microscope fields of view such that images of entire tumor cryosections were captured at a resolution of 1.5 $\mu$m/pixel.
Using both fiducial and anatomical landmarks, histological sections were chosen to match the MR slices.
Using ImageJ~\cite{Collins:2007jr} and user-supplied algorithms, digital images were superimposed and manually cropped to tumor tissue boundaries; staining artifacts and necrosis were also removed for some analyses.
Positive fluorescence for CD31 and DiOC7(3) images was obtained by applying a threshold, with neighboring positive pixels grouped as “objects.” The average distance of tissue to the nearest vascular object was reported as a repeatable measure of vascular density.
Perfused vessel fraction (PF) was calculated as the proportion of CD31-positive objects that had at least 20\% overlap with positive DiOC7(3) or HPG-GdF pixels on the overlaid image.
Data for individual tumors were displayed as mean $\pm$ SEM values.
HPG-GdF extravasation was assessed by its distance from blood vessels: pixels from the HPG-GdF fluorescence image were sorted according to their distance from vascular objects and the average HPG-GdF fluorescence intensity was reported.

\begin{figure}[htbp]   
 \begin{center}  
 \includegraphics[width=0.8\textwidth]{hpg/hpg-images/hpg_fig3-hpgdistribution.png}
 \caption{Accumulation and distribution of HPG-GdF in HT29 xenografts. (A) T1-weighted MR-RARE images show signal enhancement at 40 min that increases with longer exposures (HT29 tumors outlined in red). (B) Quantitative analysis of HPG-GdF distribution relative to CD31-positive vessels shows extravasation of HPG-GdF and distribution away from vessels to a limited distance, with a maximum at 7 days. (C) Whole tumor maps of HPG-GdF (red) in relation to vasculature (blue) show that at early (2 min) timepoints HPG-GdF is primarily overlapped with vasculature, but by 60 min there is substan- tial heterogeneity, where some vessels have greater amounts of perivascular HPG-GdF than others. HPG-GdF accumulates over several days but does so heterogeneously, and does not distribute through tumor tissue even after prolonged exposures.}  
 \label{hpgpaper:fig3}  
 \end{center}
\end{figure}

\section{Results}

\subsection{HPG-GdF accumulates in tumor tissue in the extravascular space but does not distribute far from vasculature}

Averaged data from whole HT29 tumor xenograft images obtained using MRI and histology showed accumulation of HPG-GdF over time (Fig.~\ref{hpgpaper:fig3}(A)), as previously described~\cite{Saatchi:2012hc}.
HPG-GdF fluorescence was detectable within or very near to CD31-labeled tumor vessels as early as 2 min following contrast agent injection (Fig.~\ref{hpgpaper:fig3}(B), (C)).
More HPGGdF fluorescence accumulated in histological tumor sections over time, but very little agent was observed at distances farther than 40 $\mu$m from the vasculature, even at 7 and 15 days (Fig.~\ref{hpgpaper:fig3}(B), (C)).
By 60min there was extravascular accumulation of HPG-GdF around some vessels, but considerable inter-vessel heterogeneity was observed.
Some vessels showed no HPG-GdF fluorescence (Fig.~\ref{hpgpaper:fig3}(C)).

\subsection{Bolus arrival time (BAT) for HPG-GdF as a screen for viable tissue}

Maps of BAT overlaid on T$_1$-RARE images for both Gadovist and HPG-GdF are shown in Fig.~\ref{hpgpaper:fig4}, Rows 1 and 2.
A pattern of faster contrast enhancement at the tumor margins was consistent for both contrast agents.
While Gadovist eventually distributes to all the tissue, many voxels fail to enhance with HPG-GdF within the 37 min imaging period.
Comparison of HPG-GdF BAT-enhancing voxels with a histological image delineating viable versus necrotic tissue (Fig.~\ref{hpgpaper:fig4}, Rows 2 and 3) shows that the non-enhancing voxels consistently corresponded to large areas of tumor necrosis.
A negative association was seen between the necrosis fraction determined by histology and the fraction of enhancing voxels for HPG-GdF (Fig.~\ref{hpgpaper:fig2}(C)).
For subsequent analysis of vascular function, only voxels enhancing with HPG-GdF using the BAT criteria were evaluated.

\begin{figure}[htbp]   
 \begin{center}  
 \includegraphics[width=0.8\textwidth]{hpg/hpg-images/hpg_fig4-ht29.pdf}
 \caption{Vascular function in HT29 xenografts. Whole-slice maps are presented for individual HT29 tumors (T01–T05, vertical columns) for parameters derived from MR imaging of Gadovist (BAT, AUGC60, K$^{trans}$), MR imaging of HPG-GdF (BAT, fPv, aPS, AUGC60) and histological imaging (necrosis). Good slice matching between imaging sessions is achieved using implanted fiducial marker tubes, an example of which is shown for tumor T01, Row 1, where the arrow points to the dark region where the tube is resting between the tumor and the back of the animal.}  
 \label{hpgpaper:fig4}  
 \end{center}
\end{figure}

\subsection{Fractional plasma volume (fPV) and apparent permeability–surface area product (aPS) as measures of vascular function}

HPG-GdF-enhancing voxels were further characterized for their plasma volume (fPV) by calculating the magnitude of the rapid signal increase after injection and for their extravascular accumulation (aPS) by measuring the slope of the enhancement curve after the initial increase for the duration of the DCE scan (0–2000 s).
The patterns of high and low fPV and aPS were often similar to each other, though there are notable differences.
A linear regression analysis comparing fPV with aPS for whole-slice averages yields an R$^2$ of 0.11, suggesting they are independent of each other.
As a detailed example, tumor HT01 has a region of high aPS and low fPV (Fig.~\ref{hpgpaper:fig5}(A)), as well as a region exhibiting the opposite, with high fPV and low aPS (Fig.~\ref{hpgpaper:fig5}(B)).
The corresponding histological section seemed to validate these observations, where the region with high fPV has a greater density of CD31-stained vessels and the region with greater aPS has more HPG-GdF in the extravascular compartment.
While histological data enables a detailed view of HPG-GdF accumulation, the actual rate of extravasation may only be determined by the dynamic MR data, as illustrated by the schematic enhancement curve (Fig.~\ref{hpgpaper:fig5}(C)).
The corresponding K${trans}$ map derived from Gadovist concentrations shows high values in both the high aPS and high fPV regions of this tumor (Fig.~\ref{hpgpaper:fig4}, Column 1).
Therefore, both fPV and aPS played important roles contributing to overall tumor vascular function, had measurable intra-tumor heterogeneity and produced data that was distinct from and more informative than DCE-MRI derived parameters for Gadovist.

\begin{figure}[htbp]   
 \begin{center}  
 \includegraphics[width=\textwidth]{hpg/hpg-images/hpg_fig5-ht29fpv.pdf}
 \caption{fPv and aPS in HT29 xenograft. Whole-slice parameter maps are presented for HT29 tumor (T01) with the corresponding histological image depicting CD31 stained vessels (blue) and HPG-GdF native fluorescence (red). A region with high aPS values that does not correspond to high fPV is magnified (A) and compared with a region having high fPV that does not correspond to high aPS (B). The high fPV region has notably greater vascular density, and HPG-GdF is clearly seen overlapping with vessels (black) or accumulating in the extravascular space. The high aPS region also has HPG-GdF in the extravascular space. The dynamic MR-derived parameters are better able to illustrate the functional features of tumor vasculature than are the static histological data, which only reflects the environment at a single, terminal endpoint (C).}  
 \label{hpgpaper:fig5}  
 \end{center}
\end{figure}

\begin{figure}[htbp]   
 \begin{center}  
 \includegraphics[width=\textwidth]{hpg/hpg-images/hpg_tables.pdf}
 \caption{ }  
 \label{hpgpaper:tables}  
 \end{center}
\end{figure}

\subsection{Variable vascular function in HCT116 and HT29 human colorectal xenografts}

HPG-GdF accumulated to a greater degree in HT29 colorectal xenografts relative to HCT116, and this was seen at all distances from vessels (Fig.~\ref{hpgpaper:fig6}(A)).
HCT116 and HT29 colorectal xenografts grow at similar rates in mouse models but exhibit distinct vascular function parameters (data summarized in Table 1).
Overall, HT29 tumors possessed a greater density of vessels (CD31 average distance was 82.3 $\pm$ 3.0 $\mu$m for HCT116 and 40.5 $\pm$ 1.5 $\mu$m for HT29, p < 0.05*).
However many vessels were unlabeled for the fluorescent dye (carbocyanine) used as a histological perfusion marker; the density of perfused vessels was similar between the two xenograft models (CD31 vessels labeled for carbocyanine, PF, was 47.8$\pm$5.4\% for HCT116 and 36.0$\pm$2.1\% for HT29, p > 0.05).
The density of vessels labeled for HPG-GdF fluorescence was much greater in HT29 tumors (HPG-GdF+ve CD31, average distance was 172$\pm$10$\mu$m for HCT116 and 88.9 $\pm$6.3$\mu$m; p<0.05*).
In addition to a greater density of HPGGdF-positive vessels, the high MW contrast agent was able to accumulate to a greater degree in the extravascular space around vessels in HT29 cells.
This effect can be seen in the histological images of the compared tumor models collected 60min post HPG-GdF administration (Fig.~\ref{hpgpaper:fig6}(B)).

\subsection{MRI analysis of HPG-GdF in HCT116 and HT29 xenografts: BAT, fPV and aPS}

Administration of neither Gadovist nor HPG-GdF was useful in detecting the difference in vascular function between HCT116 and HT29 tumors using the initial area under the gadolinium concentration curve (AUGC), as the slice-averaged means were not significantly different between tumor groups (AUGC for Gadovist in HCT116 was 5.09 $\pm$ 0.83 mM s and that in HT29 was 7.68 $\pm$ 1.66 mM s, p > 0.05; that for HPG-GdF in HCT116 was 6.41 $\pm$1.71×10$^{-1}$3 mMs and that in HT29 was 9.62$\pm$1.64×10$^{-1}$3 mMs, p > 0.05) (data summarized in Tables 2 and 3).
Similarity between models was also observed qualitatively in the parameter maps with AUGC or K${trans}$ overlaid on T$_1$-RARE images shown in Figs.
4 and 7.
Both tumor models show higher AUGC values in the tumor margins for both contrast agents.

\begin{figure}[htbp]   
 \begin{center}  
 \includegraphics[width=0.8\textwidth]{hpg/hpg-images/hpg_fig6-accumulation.pdf}
 \caption{HPG-GdF distribution in HCT116 and HT29 colorectal tumor xenografts at 60 min: histological analysis. (A) The whole-slice average fluores- cence intensity shows that HPG-GdF accumulates to a greater degree and distributes further away from vasculature in HT29 tumors. (B) Sample images show HPG-GdF (red) overlaid on CD31 (blue); insets also display nuclear density (grey). A greater degree of HPG-GdF accumulation can easily be ap- preciated in the HT29 tumors.}  
 \label{hpgpaper:fig6}  
 \end{center}
\end{figure}

However, an MR-measured difference between HCT116 and HT29 vascular function was seen with the BAT data for HPGGdF, as the fraction of enhancing voxels was significantly lower in the HCT116 relative to the HT29 tumors (BAT-EV for HPGGdF in HCT116 was 52.2 $\pm$ 4.8\% and that in HT29 was 81.8 $\pm$ 5.0\%, p < 0.05*) (Table 3).
This corresponded to the histological data, which also shows HCT116 tumors as having a greater proportion of necrotic tissue (necrosis fraction for HCT116 was 52.4$\pm$3.6\% versus 37.7$\pm$5.1\% in HT29, p<0.05*).
Of the HPGGdF BAT-enhancing voxels, the averaged fPV and aPS parameters were also determined (Table 3), and no difference was seen with the plasma volume between tumor models (fractional fPV for HCT116 is 2.67 $\pm$ 0.33 × 10$^{-1}$4 and that for HT29 is 2.16 $\pm$ 0.27 × 10$^{-1}$4, p > 0.05), but a trend of greater aPS was seen in the HT29 xenografts (aPS for HCT116 was 2.38 $\pm$ 1.00 × 10$^{-1}$8 and that for HT29 was 5.81 $\pm$ 1.4 × 10$^{-1}$8, p = 0.056*).
aPS of HPGGdF was the only MR-derived biomarker able to detect the significant difference in vascular function between these two tumor models.
Complete data sets for HCT116 tumors are shown in Fig.~\ref{hpgpaper:fig7}.

\section{Discussion}

The use of MCAs for imaging and describing tumor vascular function is a widely pursued area of research.
Many studies have investigated a range of sizes, where smaller molecules extravasate and distribute through tissue too quickly to be considered as sensitive blood pool agents, and larger molecules often accumulate too slowly for adequate signal detection~\cite{Kyle:2007ch,Tang:2013fi,Sourbron:2011ce}.
In this work we describe HPG-GdF as an MR-visible MCA of 583 kDa, for which we have derived useful biomarkers that have sensitivity in measuring vascular function.
While the signal-to-noise from HPG-GdF concentration–time curves is not adequate to fit a complex pharmacokinetic model to determine parameters such as Vp, Ve and K${trans}$, enhancement was measurable in most tumor voxels.
The number of Gd chelates on the HPG-GdF is approximately 300 per carrier molecule, which is high relative to many albumin chelates, the most common MCA in preclinical use, which typically have about 20 chelates per carrier~\cite{Ogan:1987tg}.
In addition to a greater number of Gd per carrier, the HPG-GdF accumulates in the perivascular space and fails to distribute more than a few micrometers from most vessels within a short DCE imaging timeframe (Fig.~\ref{hpgpaper:fig3}).
This accumulation could provide a greater concentration for detection of a local, permeable tumor vessel and avoids the risk of conflating this with the distribution and accumulation elsewhere in tumors.
These attributes of HPGGdFmay make it more sensitive and more specific than a smaller MCA such as albumin–Gd-DTPA.
A minimum DCE imaging time from the presented data would appear to be about 15 min for the analyses described, which is longer than the suggested 5min ideal for practical clinical utility~\cite{Turetschek:2004bw}.
It is possible that greater signal could also be obtained by decreasing the time resolution from the 2.4 s used in these studies, since HPG-GdF remains largely intravascular.
HPG-GdF contains the Gd-DOTA complex, which is highly thermodynamically stable and kinetically inert.
The dissociation constant for Gd-DOTA (log K = 24.7) is much higher than those for CaDOTA (log K = 17.23) or Mg-DOTA (log K = 11.92), hence neither Ca nor Mg can transmetallate Gd from the Gd-DOTA complex~\cite{Baranyai:2005ta}.
Biodegradation of HPG-GdF is therefore likely minimal, occurring through enzymatic attack of the end group, similar to that seen for polyethylene glycol~\cite{Kawai:2002fc}.
However, the long half-life and relatively slow excretion of HPGs through the reticuloendothelial system of the liver suggest that the potential for toxicity should be investigated and monitored.
A biodegradable version of HPG has recently been synthesized and might circumvent these potential concerns, and will be a focus in future DCE studies investigating the use of HPGs as MR-visible MCAs~\cite{Shenoi:2013id}.
Limited extravasation of HPG-GdF is likely due to the selective permeability of tumor vessels with adequate pore sizes for passage of the bulky molecule, a phenomenon described as the enhanced permeability and retention (EPR) effect~\cite{Maeda:2013hq}.
Limited distribution through the interstitium is also most probably the result of its significant size.
We have observed HPG-GdF to be much more restricted in its distribution than is suggested by the diffusion coefficients (D(HPG) = 3.7 × 10$^{-1}$7 cm$^2$ s$^{-1}$1 and D(Gadovist) = 3.6 × 10$^{-1}$6 cm$^2$ s$^{-1}$1), as the larger agent reaches only 10–15$\mu$m away from vessels within an hour whereas Gadovist reaches all areas of a tumor.
HPG-GdF molecules experience greater obstacles to their movement through the heterogeneous tumor microenvironment due to their size, but also possibly due to steric hindrances, non-specific binding or sequestration~\cite{Minchinton:2006gs}.
The large size of HPG-GdF is a significant advantage.
An approximately exponential increase in sensitivity for detection of permeability has been reported with increasing MCA size (19).
If extravasation of HPG-GdF only occurs in vessels that are permeable, and the agent does not distribute through the extravascular space to distances away from vessels, then MR-visible HPG-GdF enhancement suggests the presence of vasculature, and accumulation of the agent over time indicates the local presence of a hyperpermeable vessel.

\begin{figure}[htbp]   
 \begin{center}  
 \includegraphics[width=0.8\textwidth]{hpg/hpg-images/hpg_fig7-hct116.pdf}
 \caption{Vascular function in HCT116 xenografts. Whole-slice maps are presented for individual HCT116 tumors (T01-T05, vertical columns) for parameters derived from MR imaging of Gadovist (BAT, AUGC60, K$^{trans}$), MR imaging of HPG-GdF (BAT, fPv, aPS, AUGC60) and histological imaging (necrosis).}
 \label{hpgpaper:fig7}  
 \end{center}
\end{figure}

A typical enhancement curve for HPG-GdF is shown in Fig.~\ref{hpgpaper:fig2}.
The size of the initial step-like enhancement is interpreted as the fPV, and the slope of signal change for the remaining enhancement is reported as the aPS, reflecting the extravascular accumulation of HPG-GdF (Fig.~\ref{hpgpaper:fig2}).
These parameters are not correlated with each other, with either value potentially significantly higher or lower than the other in the same voxels (Fig.~\ref{hpgpaper:fig5}).
A pattern of faster enhancement at the tumor margins is consistent for both contrast agents, and, while Gadovist eventually arrives in all the tissue, including regions of necrosis, many voxels that correspond to necrotic regions fail to enhance with HPG-GdF within the 37 min imaging session.
We found that HPG-GdF BAT-enhancing voxels correspond to areas without necrosis: see Figs.~\ref{hpgpaper:fig4} and ~\ref{hpgpaper:fig7} for parameter maps of all BAT-enhancing voxels compared with histologically validated necrotic regions.
Therefore, we identified perfused regions of tumor tissue for further vascular function analysis using the straightforward approach of selecting voxels that had positive BATs for their HPG-GdF enhancement curves.
Selection of viable tissue and the exclusion of necrotic tissue is a desirable approach for controlling for inter-tumor heterogeneity, since MRI data is most often described as a whole-tumor average.
In the event that different tumors possess variable necrotic burdens, or if the necrotic fraction increases over time with treatment, selective analysis of viable tissue can control for this variability.
Qualitative comparisons of well-matched, whole-slice images from multiple modalities emphasize the utility of screening for BAT-enhancing voxels to select for perfused and viable tissues.

Without this information analysis may be restricted to regions of tissues at the tumor margins where hot spots of vascularization are observed histologically, and tissue is assumed to be viable in the MR image~\cite{Pathak:2005gu,Li:2005gw}.
In the colorectal xenografts examined here the amount of necrosis varied considerably, and, while it is typically located more in the core than the tumor margins, there is substantial inter-tumor heterogeneity with respect to the amount and location of necrosis.
Notably, while the values for K${trans}$(Gadovist) are on average higher around the entire tumor rim, the fPV and aPS values are more heterogeneously arranged around the rim and within the interior of the tumor.
Our more comprehensive approach, validated by observations of viable tissue and perfused vessels within tumor interiors in histological sections, is a significant improvement over selectively assessing limited regions or hotpots for histological or MR quantitative analysis.
Traditional DCE-derived parameters such as AUC and K${trans}$ are composite measures influenced to varying degrees by vascular surface area, permeability and blood flow.
Our histological data supports a significant range in the propensity for HPG-GdF to extravasate, suggesting that assumptions of perfusionor permeability-limited conditions are not applicable to all areas of the tumor microenvironment.
Hence, we cannot conclude that aPS is exclusively proportional to permeability– surface area (PS) in the whole microenvironment.
For example, although the amount of HPG-GdF leaking into the extravascular space is dependent on the vessel being permeable to large molecules, the amount of accumulation, and therefore the amount of signal enhancement, may still be impacted by the concentration of contrast agent within the vessel.
Longitudinal gradients can occur locally in permeable vessels as the contrast agent leaks out, despite plasma concentrations remaining consistent in overall systemic circulation~\cite{Erickson:2003wt,Dewhirst:1999jh}.
Thus, the aPS is a measure of vascular function that has contributions from permeability as well as, when permeability is very high, perfusion.
These limitations in the physiological interpretation of aPS are similar to those that apply to K${trans}$ and AUC for all contrast-enhanced modeling.
Parameters that are based on the amount and proportion of contrast agent enhancement are dependent on fewer assumptions than are pharmacokinetic model-derived parameters such as K${trans}$.
Biophysical signals are dependent on the many variables necessarily involved in obtaining MR images, such as the scanner, imaging sequence, RF coil and analysis techniques.
Interpreting magnitudes of change in highly variable tumors may be more reliable than assuming invariable pharmacokinetic attributes or unchanging tumor microenvironments, and may make simplified biomarkers such as fPV and aPS more applicable to clinical studies conducted across multiple centers~\cite{OConnor:2012ie}

\section{Conclusion}

HPG-GdF is a largely intravascular MCA that selectively extravasates from hyperpermeable tumor vessels, accumulating in the perivascular regions without distributing through the tumor interstitium.
The high concentration of Gd chelates per carrier molecule in combination with its excellent solubility makes HPG-GdF detection possible despite the relatively low plasma fraction within tumors.
By carefully comparing the vessel parameters of small and large molecule contrast agents in the same tumor, as well as comprehensively assessing the location of MCA within the tumor relative to vasculature and necrosis, we conclude that BAT, fPV and aPS biomarkers derived from HPG-GdF enhancement provide a sensitive and specific approach to measuring tumor vascular function.
HPG-GdF and these analysis techniques would be appropriate in the evaluation of tumor angiogenesis and response to treatment for preclinical research, with potential for eventual translation to the clinic.

\endinput

Any text after an \endinput is ignored.
You could put scraps here or things in progress.
