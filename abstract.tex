%% The following is a directive for TeXShop to indicate the main file
%%!TEX root = diss.tex

\chapter{Abstract}

This thesis comprises development and application of several MRI techniques to improve our understanding of tumour growth, drug distribution, and drug effect using pre-clinical tumour models in mice. In the first part of the thesis, a novel high molecular weight contrast agent, HPG-GdF is introduced. This molecule is a hyperbranched polyglycerol labeled with an MRI contrast agent (Gd-DOTA) as well as a fluorescent tag. After injecting the agent into mice within an MRI scanner, contrast-agent kinetics were quantified using a two-parameter linear model and validated with quantitative immunohistochemistry via direct fluorescence imaging of HPG-GdF.

HPG-GdF was used to assess whether vascular function plays a role in how a chemotherapy (Herceptin) distributes within a tumour. Tumour vessel permeability and fractional plasma volume were quantified using the HPG-GdF and no relationship was found between vascular function and presence of drug. HPG-GdF was then applied to show that Avastin (an antiangiogenic agent) decreased vessel permeability in tumours. Using histological methods, a dramatic reduction in hypoxia (oxygen deficiency in tissues) was observed in treated tumours. Unfortunately, existing MRI methods to evaluate oxygenation were time-intensive and lacked sensitivity. In the second part of this thesis, we introduce, develop, validate, and apply a new method to assess tumour oxygenation using MRI. 

Oxygen (O$_2$) is a paramagnetic molecule that shortens the longitudinal relaxation time (T$_1$) of protons in MRI. This subtle effect has been widely reported in the literature but its applications in cancer have been limited. Our technique - dynamic oxygen-enhanced MRI (\acs{dOE-MRI}) - uses T$_1$W signal intensity images acquired during a cycling gas challenge (air or oxygen) and independent component analysis (ICA). Hypoxia staining with pimonidazole correlated strongly with \acs{dOE-MRI} values in a murine tumour model (SCCVII) and only weakly in a colorectal xenograft model (HCT-116). Finally, we provide compelling evidence that treatment with Avastin improves tumour oxygenation in subcutaneous tumours. With \acs{dOE-MRI}, the sensitivity and speed of existing techniques was greatly improved. Since our technique requires no injectable contrast agent, special sequences or hardware, we anticipate that this technique can be quickly translated into the clinic. 


% Consider placing version information if you circulate multiple drafts
%\vfill
%\begin{center}
%\begin{sf}
%\fbox{Git commit:}
%\end{sf}
%\end{center}
