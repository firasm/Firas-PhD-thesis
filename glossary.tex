%% The following is a directive for TeXShop to indicate the main file
%%!TEX root = diss.tex

\chapter{Glossary}

% use \acrodef to define an acronym, but no listing
%\acrodef{UI}{user interface}
%\acrodef{UBC}{University of British Columbia}

%\acrodef{DCE-US}{dynamic contrast-enhanced ultrasound}
%\acrodef{ICA}{Independent Component Analysis}


% The acronym environment will typeset only those acronyms that were
% *actually used* in the course of the document
\begin{acronym}

\acro{ICA}[ICA]{Independent Component Analysis}%; a blind source separation technique that splits a multi-source signal into its constituents}

\acro{IVA}[IVA]{IVA Independent vector analysis}

\acro{FastICA}[FastICA]{FastICA -  An ICA algorithm implemented in the \texttt{scipy.sklearn v0.17.1} python package}

\acro{dOE-MRI}[dOE-MRI]{dynamic Oxygen Enhanced MRI}%; a technique that relies on the T$_1$-shortening property of oxygen and independent component analysis to identify regions that respond to an oxygen stimulus.}

\acro{DCE-MRI}[DCE-MRI]{dynamic contrast-enhanced MRI}

\acs{DTPA}[DTPA]{DTPA}

\acs{DOTA}[DOTA]{DOTA}

\acro{DSC-MRI}[DSC-MRI]{dynamic susceptibility weighted MRI}

\acro{ASLI}[ASL]{arterial spin labeling}

\acro{DCE-US}[DCE-US]{dynamic contrast-enhanced ultrasound}%; an imaging technique that uses echogenic microbubbles to enhance contrast \emph{in vivo}.}

\acro{CD31}[CD31]{cluster of differentiation 31}%, also known as platelet endothelial cell adhesion molecule (PECAM) is a protein that is used in immunohistochemistry primarily to demonstrate the presence of endothelial cells.}

\acro{Hoechst 33342}[bisbenzimide]{Hoechst 33342 (also known as bisbenzimide) is an organic compound used as a fluorescent stain for DNA}

\todo[backgroundcolor=blue]{Really... EPR is EPR..}
\acro{EPR}[EPR]{EPR}%, or enhanced permeability and retention is a phenomenon that results in macromolecules are retained in the tumour and extravasation into the tumour interstitium continues due to poor lymphatic clearance and other factors.}

\acro{CA}[CA]{Contrast agent}

\acro{BAT}[BAT]{BAT Bolus arrival time}

\acro{kDa}[kDa]{kDa kilo dalton, or 1000 daltons}

\acro{Da}[Da]{Da Dalton, a unit of measurement for molecular weight}

\acro{EPR}[EPR]{EPR Enhanced permeability retention effect}

\acro{MCA}[MCA]{Macromolecular contrast agents.}

\acro{aPS}[aPS]{Apparent permeability surface area product}%; a parameter derived from DCE-MRI using HPG. - a parameter derived from a linear model describing the contrast agent kinetics of a high molecular weight contrast agent}

\acro{fPV}[fPV]{Fractional plasma volume}

\acro{HPG-GdF}[HPG-GdF]{Hyperbranched polyglycerol (HPG)}% chelated to Gd-DTPA with a fluorophore (F) attached; 500 kDa}

\acro{Gd-DTPA}[Gd-DTPA]{Gadolinium chelated to (Diethylenetriamine Pentaacetic Acid)}%is a strongly paramagnetic chemical element in the Lanthanide series on the periodic table. It is attached to a chelating agent such as DTPA (Diethylenetriamine Pentaacetic Acid) to make it biocompatible. Upon administration of Gd-DTPA through the blood stream, T$_1$ is drastically shortened and the agent is visible on MR images.}

\acro{MW}[MW]{Molecular weight}

\acro{VEGF}[VEGF]{Vascular endothelial growth factor}

\acro{fPV}[fPV]{Fractional plasma volume (fPV)}

\acro{AUC}[AUC]{Area under the Curve}%; a parameter that is often calculated for contrast-kinetics curves in DCE-MRI. AUC$_{60}$ is a typically reported parameter and indicates the area under the curve 60 seconds after injection.}
\acro{MAb}[MAb]{Monoclonal antibody}

\acro{MAbs}[MAbs]{Monoclonal antibodies}

\acro{ASL}[ASL]{Arterial Spin Labelling}

\acro{AIF}[AIF]{AIF arterial input function}

\acro{$v_e$}[$v_e$]{$v_e$ Volume of extravascular extracellular space per unit volume of tissue}

\acro{$v_p$}[$v_p$]{$v_p$ Blood plasma volume per unit volume of tissue}

\acro{K$^{trans}$}[K$^{trans}$]{K$^{trans}$ Volume transfer constant}

\acro{K$_{ep}$}[K$_{ep}$]{K$_{ep}$}

\acro{HER2}[HER2]{HER2 stands or human epidermal growth factor receptor 2}

\acro{BBB}[BBB]{Blood brain barrier}

\acro{i.p.}[i.p.]{Intraperitoneal injections take place in the body cavity of animals}

\acro{i.v.}[i.v.]{Intravenous injections are administered through a vein of an animal}

\acro{i.c.}[i.c.]{Intracardiac injections are administered directly into the heart muscle or ventricles of an animal}

\acro{BT474}[BT474]{Cell line derived from a human mammary gland (breast/duct) cancer}

\acro{MDA-MB-361}[MDA-MB-361]{Cell line derived from a breast cancer that metastasized to the brain}

\acro{RARE}[RARE]{RARE sequence is a rapid acquisition with refocused echoes. This sequence is also known known as fast spin echo (FSE) or turbo spin echo (TSE)}

\acro{NCWF}[NCWF]{NCWF Normalized component weighting factor value}

\acro{IM}[IM]{IM intramuscular}

\acro{SC}[SC]{SC subcutaneous}

\acro{DCE-US}[DCE-US]{DCE-US dynamic contrast-enhanced ultrasound}

\acro{USPIO}[USPIO]{USPIO ultrasmall superparamagnetic iron oxide}

\acro{ROI}[ROI]{ROI region of interest}

\acro{MCA}[MCA]{MCA Macromolecular contrast agent}

\acro{CA}[CA]{CA contrast agent}

\acs{Gd-DOTA}[Gd-DOTA]{Gd-DOTA}

\acs{MW}[MW]{MW Molecular weight}

\acs{MGE}[MGE]{MGE multi gradient echo}

\acs{SNR}[SNR]{SNR Signal to noise ratio}

\end{acronym}

% You can also use \newacro{}{} to only define acronyms
% but without explictly creating a glossary
% 
% \newacro{ANOVA}[ANOVA]{Analysis of Variance\acroextra{, a set of
%   statistical techniques to identify sources of variability between groups.}}
% \newacro{API}[API]{application programming interface}
% \newacro{GOMS}[GOMS]{Goals, Operators, Methods, and Selection\acroextra{,
%   a framework for usability analysis.}}
% \newacro{TLX}[TLX]{Task Load Index\acroextra{, an instrument for gauging
%   the subjective mental workload experienced by a human in performing
%   a task.}}
% \newacro{UI}[UI]{user interface}
% \newacro{UML}[UML]{Unified Modelling Language}
% \newacro{W3C}[W3C]{World Wide Web Consortium}
% \newacro{XML}[XML]{Extensible Markup Language}
