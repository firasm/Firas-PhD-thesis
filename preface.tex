%% The following is a directive for TeXShop to indicate the main file
%%!TEX root = diss.tex

\chapter{Preface}
\label{ch:preface}

The work presented in this thesis relies on development of MRI sequences, pre-clinical animal experiments, immunohistological staining and analysis, as well as post-acquisition MRI data analyses in collaboration with multiple researchers.
Since different researchers played a part in different aspects of the works, the preface is divided by chapter to ensure adequate credit is provided to the people responsible.
In general, Dr.\ Jennifer Baker working in the lab of Dr.\ Andrew Minchinton at the BC Cancer Research centre provided all of the tumour xenograft bearing mice used in experiments, and contributed extensive biological expertise and knowledge to the works presented in this thesis.
All animal experimental procedures in this thesis were carried out in compliance with the guidelines of the Canadian Council for Animal Care and were approved by the institutional Animal Care Committee.

%\subsubsection{Chemical Exchange Saturation Transfer (CEST) - Chapter~\ref{ch:CEST}}
%
%The author acquired and analyzed all of the MRI and histology data for this section of the thesis.
%Dr.\ Baker, assisted with the tumour treatments, as well as the histology analysis.
%Dr.\ Reinsberg provided much guidance and support in the design of the experiment, development of the sequence, and in troubleshooting the myriad problems that arose during sequence development. 
%Mr. Andrew Yung also provided invaluable support in developing and troubleshooting the CEST sequence.
%The initial interest in CEST-MRI was a carry-over from a side project the author participated in during his masters degree at the University of Toronto with Drs. Kim Desmond and Greg J. Stanisz. 
%Dr.\ Stanisz was kind enough to provide the initial CEST sequence as well as MATLAB code to analyze the data which was used as the baseline to compare our improved technique against.

\subsubsection{Chapter~\ref{ch:HPG} - \nameref{ch:HPG}}

Work presented in chapter~\ref{ch:HPG} has been published in its entirety in \emph{Contrast Media \& Molecular Imaging} with the manuscript titled ``Multi-modal magnetic resonance imaging and histology of vascular function in xenografts using macromolecular contrast agent hyperbranched polyglycerol (\acs{HPG-GdF})''~\cite{Baker:2015cob}.
The author of this thesis is listed as the 3$^{rd}$ author on this manuscript.

Drs. Baker and Reinsberg initiated a collaboration with Drs. Katayoun Saatchi and Urs Hafeli from the Faculty of Pharmaceutical Sciences.
Drs. Saatchi and Hafeli had synthesized a new hyper-branched poly-glycerol molecule (\acs{HPG-GdF}) that was biocompatible and could be used as a high molecular weight MRI contrast agent.
During her masters degree at UBC, Dr.\ Kelly McPhee began work on this project to characterize the molecule. 
She performed all pilot and phantom experiments to test and measure its relaxivity, as well as initial pilot testing, QA to assess the viability of the molecule as an MRI contrast agent, as well as acquisition and analysis of pilot animal data.
The work described above is not a part of this thesis, but the development and validation was essential and provided a springboard for this author's work.

Following initial characterization of the molecule, in collaboration with Dr.\ Baker and Dr.\ Reinsberg, Dr.\ McPhee also acquired and analysed animal data from 10 mice. 
Raw and some processed MRI data (notably the calculated concentration-time curves) from this experiment were used in this manuscript by this author.
The raw data was used to develop and present a new method to quantify \acs{HPG-GdF} concentration-time curves using a two-parameter linear model.
To develop this method, several follow-up experiments with a new MRI sequence and improved analysis methods was done in collaboration with Drs. Baker and Reinsberg - this data is not part of this chapter but is instead presented in Chapters~\ref{ch:HPG2} and ~\ref{ch:HPG3}.
Dr.\ Reinsberg provided much guidance and support in the design of the experiment, development of the sequence, and in troubleshooting the myriad problems that arose during sequence development.

Initial draft of the manuscript was prepared primarily by Dr.\ Baker based on the final chapter of Dr.\ McPhee's MSc. thesis, with this author contributing to the new MRI methods, followed by the MRI analysis, as well as the results and discussion sections. 
In developing this manuscript, all analysis code was written by this author and all parameteric maps were generated using custom code written in Python.
Code to read in generic MRI data was developed by Dr.\ Reinsberg with some assistance from Andrew Yung, a research scientist at the 7T research centre.
All histology data presented in this paper was collected and processed by Dr.\ McPhee under the supervision of Dr.\ Baker, and all final figure preparation including manuscript submission was done by Dr.\ Baker.
Reviewer comments were addressed collaboratively with Dr.\ McPhee, Dr.\ Baker, Dr.\ Reinsberg, and this author equally.

\subsubsection{Chapter~\ref{ch:HPG2} - \nameref{ch:HPG2}}

Work presented in chapter~\ref{ch:HPG2} has been published in its entirety in \emph{Clinical \& Experimental Metastasis} with the manuscript titled ``Heterogeneous distribution of trastuzumab in HER2-positive xenografts and metastases: role of the tumor microenvironment''~\cite{Baker:2018ex}.

The author of this thesis is listed as the 4$^{th}$ author on this manuscript.
As this manuscript was part of a much larger study led by Dr.\ Baker, only the sections that pertain to MRI data and analysis have been reproduced here.
This author contributed to the experimental design of the study, collected and analysed all MRI data presented in this study, and assisted with figure preparation.
Dr.\ Baker wrote the first draft of this manuscript excluding the MRI methods and results, with Dr.\ Reinsberg and this author assisting with editing and writing of the methods, results, and discussion sections that pertained to MRI.

\subsubsection{Chapter~\ref{ch:HPG3} - \nameref{ch:HPG3}}

Work presented in chapter~\ref{ch:HPG3} will be submitted to the International Society for Magnetic Resonance in Imaging (ISMRM) conference in November of 2019, with this author as the lead author.

Dr.\ Baker had the initial idea to use \acs{HPG-GdF} as a macromolecular contrast agent to assess response of anti-angiogenic agents, designed the experiments, prepared the tumours, and conducted the interventions.
Dr.\ Baker performed all the histological analysis including sectioning, staining, imaging, and cropping of the tumours.
This author contributed to the experimental design of the study, collected and analysed all MRI data presented in this study, generated all the figures, and wrote all of the text. 
Dr.\ Baker and Dr.\ Reinsberg were present during the imaging data collection.
Interpretation of results - particularly of histological images - was done collaboratively by this author, Dr.\ Reinsberg, and Dr.\ Baker.

\subsubsection{Chapter~\ref{ch:oemri1} - \nameref{ch:oemri1}}

With the exception of Sections~\ref{sec:numComponents},~\ref{sec:correctionfactor}, and \ref{sec:interleave} material presented in this chapter was published in the \textit{Journal of Magnetic Resonance in Medicine}~\cite{Moosvi:2018ca}. 

This author was the lead investigator, responsible for all major areas of MRI data collection, analysis, manuscript composition. 
Dr.\ Baker initially approached us with the biological need to assess tumour hypoxia non-invasively using MRI and completed the immunohistochemistry staining and analysis for this project. 
Dr.\ Reinsberg was the supervisory author on this project and was involved throughout the project in concept formation and manuscript composition; he also came up with the initial idea to apply un-supervised machine learning techniques to our data. 
Dr.\ Martin McKeown provided assistance in understanding the utility of \acs{ICA} in the given context. 

\subsubsection{Chapter~\ref{ch:oemri2} - \nameref{ch:oemri2}}
Sections \ref{sec:rangeModels}, \ref{sec:histoSections} have also been published as part of the first OE-MRI manuscript~\cite{Moosvi:2018ca}.
Sections \ref{sec:lognormalfitting_theory}, \ref{sec:lognormalfitting_methods}, and \ref{sec:lognormalfitting_results} pertain to work that was completed after the first publication and will be combined with the work in Chapter~\ref{ch:oemri3} for a new manuscript to be submitted to \textit{NMR in Biomedicine} with this author as principal author.

\subsubsection{Chapter~\ref{ch:oemri3} - \nameref{ch:oemri3}}

This work was done in collaboration with Dr.\ Baker.
This author responsible for all aspects of MRI data collection, analysis, interpretation, manuscript composition. 
Dr.\ Baker's initially expressed the need to assess tumour oxygenation after modulating it with an anti-angiogenic compound.
Dr.\ Baker also processed the immunohistochemistry staining, produced the images, proposed the drug, and we collaboratively designed the experiments. 
Dr.\ Reinsberg was the supervisory author on this project and was involved throughout the project in concept formation and manuscript composition. 
Both Drs. Baker and Reinsberg were instrumental in editing of the manuscript and providing guidance on data presentation, visualization.
A version of this chapter will be submitted as a manuscript to \textit{NMR in Biomedicine} as a stand-alone MRI intervention paper.
Portions of this chapter are also being prepared as part of a larger body of work on increasing radiation sensitivity of tumours using anti-angiogenic agents with Dr.\ Baker as principal author.

\subsubsection{Chapter~\ref{ch:futurework} - \nameref{ch:futurework}}

All analysis presented in this chapter was conducted solely by this author and contributions for the data collected has already been reported in previous chapters.
