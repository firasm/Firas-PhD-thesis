%% The following is a directive for TeXShop to indicate the main file
%%!TEX root = diss.tex

\chapter{Preface}
\label{ch:preface}
The work presented in this thesis comprises development of MRI sequences, pre-clinical animal experiments, immunohistological staining and analysis, as well as post-acquisition MRI data analyses in collaboration with multiple researchers.
Since different researchers played a part in different aspects of the works, the preface is divided into three subsections, one for each of the main bodies of work.
In general, Dr. Jennifer Baker working in the lab of Dr. Andrew Minchinton at the BC Cancer Research centre provided all of the tumour xenograft bearing mice used in experiments, and contributed extensive biological expertise and knowledge to the works presented in this thesis.
The author's research supervisor Dr. Stefan Reinsberg has been instrumental in providing constant guidance, encouragement and made innumerable intellectual contributions to support and advance research in all areas.

%\subsubsection{Chemical Exchange Saturation Transfer (CEST) - Chapter~\ref{ch:CEST}}
%
%The author acquired and analyzed all of the MRI and histology data for this section of the thesis.
%Dr. Baker, assisted with the tumour treatments, as well as the histology analysis.
%Dr. Reinsberg provided much guidance and support in the design of the experiment, development of the sequence, and in troubleshooting the myriad problems that arose during sequence development. 
%Mr. Andrew Yung also provided invaluable support in developing and troubleshooting the CEST sequence.
%The initial interest in CEST-MRI was a carry-over from a side project the author participated in during his masters degree at the University of Toronto with Drs. Kim Desmond and Greg J. Stanisz. 
%Dr. Stanisz was kind enough to provide the initial CEST sequence as well as MATLAB code to analyze the data which was used as the baseline to compare our improved technique against.

\subsubsection{HPG - Chapters~\ref{ch:HPG} and \ref{HPG2}}

I acquired most of the MRI data presented here and analyzed all of it.
Drs. Baker and Reinsberg initiated a collaboration with Drs. Katayoun Saatchi and Urs Hafeli from the Faculty of Pharmaceutical Sciences.
Drs. Saatchi and Hafeli had synthesized a new hyper-branched poly-glycerol molecule that was biocompatible and could be used as a high molecular weight MRI contrast agent.
During her masters degree at UBC, Dr. Kelly McPhee began work on this project doing initial pilot testing, QA to assess the viability of the molecule as an MRI contrast agent, as well as a acquisition and analysis of an MRI experiment.
In collaboration with Dr. Baker, Dr. McPhee also acquired and analyzed MRI data for an experiment as part of her masters thesis. 
A portion of the data acquired and some of the analysis for the publication presented in ChapterXX~\todo{get the author label} was done by Dr. McPhee.
I conducted several follow-up experiments with a new MRI sequence and improved analysis methods in collaboration with Drs. Baker and Reinsberg. 
Dr.  Reinsberg provided much guidance and support in the design of the experiment, development of the sequence, and in troubleshooting the myriad problems that arose during sequence development.
The HPG project was a fairly large project and the author made major contributions to both works that have so far been published, as well as several experiments that will be published soon.

\subsubsection{Oxygen-enhanced MRI (Chapter~\ref{ch:oemri1})}

With the exception of Sections~\ref{sec:numComponents},~\ref{sec:correctionfactor}, and \ref{sec:interleave} material presented in this chapter was published in the \textit{Journal of Magnetic Resonance in Medicine}~\cite{Moosvi:2018ca}. 
I was the lead investigator, responsible for all major areas of MRI data collection, analysis, manuscript composition. 
Dr. Baker initially approached us with the biological need to assess tumour hypoxia non-invasively using MRI and this prompted the lab delving into oxygen enhanced MRI; completed the immunohistochemistry staining and analysis for this project.
Dr. Reinsberg was the supervisory author on this project and was involved throughout the project in concept formation and manuscript composition; he also came up with the initial idea to apply un-supervised machine learning techniques to our data. 
Both Drs. Baker and Reinsberg were instrumental in editing of the manuscript and providing guidance on data presentation, visualization.
Dr. Martin McKeown provided assistance in understanding the utility of \ac{ICA} in the given context. 
Dr. Alastair Kyle undertook foundational work by designing and building the microscope and histology acquisition system. 
All experiments presented here were approved by the University of British Columbia (UBC) Animal Care Committee (ACC).

\subsubsection{Validating dynamic Oxygen Enhanced MRI (Chapter~\ref{ch:oemri2})}
Sections \ref{sec:rangeModels},\ref{sec:histoSections} have also been published as part of the first OE-MRI manuscript~\cite{Moosvi:2018ca}.
Sections \ref{sec:lognormalfitting_theory}, \ref{sec:lognormalfitting_methods}, and \ref{sec:lognormalfitting_results} pertain to work that was completed after the first publication and will be combined with the work in Chapter~\ref{ch:oemri3} for a new manuscript to be submitted to \textit{NMR in Biomedicine} with me as lead author.
The remaining sections of this chapter are unpublished and will likely not be published due to their ``QA/QI'' (quality assurance and improvement) nature.

\subsubsection{Applying dynamic Oxygen Enhanced MRI (Chapter~\ref{ch:oemri3})}
This work was done in collaboration with Dr. Baker.
I was responsible for all aspects of MRI data collection, analysis, interpretation, manuscript composition. 
It was Dr. Baker's hypothesis that dOE-MRI could be used to assess tumour oxygenation after modulating it with an antiangiogenic compound. 
Dr. Baker processed the immunohistochemistry staining, provided histological images, proposed the drug as well as its mechanism of action, and we collaboratively designed the experiments. 
Dr. Reinsberg was the supervisory author on this project and was involved throughout the project in concept formation and manuscript composition. 
Both Drs. Baker and Reinsberg were instrumental in editing of the manuscript and providing guidance on data presentation, visualization.
A version of this chapter will be submitted as a manuscript to \textit{NMR in Biomedicine} as a stand-alone MRI intervention paper, with me as lead author.
Portions of this chapter may also be included as part of a larger body of work on increasing radiation sensitivity of tumours using anti-angiogenic agents with Dr. Baker as principal author.
All experiments presented here were approved by the University of British Columbia (UBC) Animal Care Committee (ACC).