%% The following is a directive for TeXShop to indicate the main file
%%!TEX root = diss.tex

\chapter{Abstract}

This thesis comprises several MRI techniques to improve our understanding of tumour growth, drug distribution, and drug effect using pre-clinical tumour models in mice. In the first part of the thesis, we introduce a novel high molecular weight contrast agent: hyperbranched polyglycerol labeled with an MRI contrast agent (Gd-DOTA) as well as a fluorescent tag: HPG-GdF.
Its kinetics were quantified using a linear model and we validated the MRI parameters with quantitative immunohistochemistry by directly staining for HPG-GdF.

HPG-GdF was used to assess whether vascular function explains how a chemotherapy distributes within a tumour. Using parameters obtained from the linear model, we quantified vessel permeability and fractional plasma volume. We then applied HPG-GdF to show that Avastin (an antiangiogenic agent) decreased vessel permeability in tumours. Using histological methods, a dramatic reduction in hypoxia (oxygen deficiency in tissues) was observed in treated tumours. Unfortunately, existing methods to measure oxygenation in tumours using MRI lacked sensitivity and specificity. 

In the second part of this thesis, we introduce, develop, validate, and apply a new method to assess tumour oxygenation using MRI. Oxygen (O$_2$) is a paramagnetic molecule and shortens the longitudinal relaxation time (T$_1$) of protons in MRI. This subtle but measurable influence of tissue oxygenation on T$_1$ in tumours has been reported in the literature but its applications in cancer have been limited. Using a cycling gas challenge (either air or oxygen) and \acs{ICA}, we greatly improved the sensitivity and speed of existing techniques. Changes in oxygenation were detected using dynamically-acquired T$_1$W signal intensity images: \acs{dOE-MRI}. Hypoxia staining with pimonidazole correlated strongly with \acs{dOE-MRI} in a murine tumour model (SCCVII). Finally, we conclusively showed that treatment with Avastin improves tumour oxygenation. The treatment effect was observed at both 24 and 48 hours after treatment. Since \acs{dOE-MRI} requires no injectable contrast agent, special sequences, or hardware requirements, it can be quickly translated into the clinic with minimal modifications. We anticipate that this technique 
\SARi{expand a bit on your findings and achievements with dOE-MRI}

% Consider placing version information if you circulate multiple drafts
%\vfill
%\begin{center}
%\begin{sf}
%\fbox{Revision: \today}
%\end{sf}
%\end{center}
