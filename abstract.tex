%% The following is a directive for TeXShop to indicate the main file
%%!TEX root = diss.tex

\chapter{Abstract}

This thesis comprises several MRI techniques to improve our understanding of tumour growth, drug distribution, and drug effect using pre-clinical tumour models in mice. In the first part of the thesis, a novel contrast agent is introduced and its kinetics quantified using a linear model. With our collaborators from the BC Cancer Research Centre, we validated the MRI parameters with quantitative immunohistochemistry by directly staining for the new agent.

We first used the agent to assess whether vascular function can help explain how a chemotherapy (Herceptin) distributes within a tumour. Microregional patterns of Herceptin distribution in vivo did not consistently correlate with vascular density, patency, function or maturity; areas of poor drug access were not necessarily those with poor vascular supply. These data directly demonstrate tissue- and vessel-level barriers to Herceptin distribution that can effectively limit access of the drug to target cells. We also described a second application of this contrast agent: we showed that Avastin - an antiangiogenic agent - decreased vessel permeability of tumours. We observed an unexpected result after treating tumours with Avastin: histology staining of tumour hypoxia (oxygen deficiency in tissues) was dramatically reduced in treated tumours. 

Unfortunately, existing methods to measure oxygenation in tumours using MRI lacked sensitivity and specificity. There is a critical need for non-invasive imaging biomarkers of tumour oxygenation to assist in patient stratification and development of hypoxia targeting therapies.

In the second part of this thesis, we introduce, develop, validate and apply a new method to assess tumour oxygenation using MRI. Using a cycling gas challenge (either air or oxygen) and independent component analysis (\acs{ICA}), we sought to improve the sensitivity and speed of existing oxygen enhanced MRI techniques to detect changes in oxygenation with dynamically acquired T$_1$W signal intensity images (dynamic oxygen-enhanced MRI, \acs{dOE-MRI}). We used the technique and validated it with histology to conclusively show that treatment with Avastin improves tumour oxygenation. Since \acs{dOE-MRI} requires no injectable contrast agent or special sequences or hardware requirements, it has the potential to be quickly translated into the clinic.

% Consider placing version information if you circulate multiple drafts
%\vfill
%\begin{center}
%\begin{sf}
%\fbox{Revision: \today}
%\end{sf}
%\end{center}
