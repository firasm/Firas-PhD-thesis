%% The following is a directive for TeXShop to indicate the main file
%%!TEX root = diss.tex

\chapter{Abstract}

This thesis comprises several MRI techniques to improve our understanding of tumour growth, drug distribution, and drug effect using pre-clinical tumour models in mice. In the first part of the thesis, we introduce a novel high molecular weight contrast agent HPG-GdF. This molecule is a hyperbranched polyglycerol labeled with an MRI contrast agent (Gd-DOTA) as well as a fluorescent tag. After injecting the agent in the MRI scanner, we quantified its kinetics using a linear model and validated the parameters with quantitative immunohistochemistry by directly staining for HPG-GdF.

HPG-GdF was used to assess whether vascular function explains how a chemotherapy distributes within a tumour. Using parameters obtained from the linear model, we quantified vessel permeability and fractional plasma volume. We then applied HPG-GdF to show that Avastin (an antiangiogenic agent) decreased vessel permeability in tumours. Using histological methods, a dramatic reduction in hypoxia (oxygen deficiency in tissues) was observed in treated tumours. Unfortunately, existing methods to measure oxygenation in tumours using MRI were time-intensive and lacked sensitivity. In the second part of this thesis, we introduce, develop, validate, and apply a new method to assess tumour oxygenation using MRI. 

Oxygen (O$_2$) is a paramagnetic molecule and shortens the longitudinal relaxation time (T$_1$) of protons in MRI. This subtle effect has been widely reported in the literature but its applications in cancer have been limited. Our technique - dynamic oxygen-enhanced MRI (\acs{dOE-MRI}) - uses T$_1$W signal intensity images acquired dynamically during a cycling gas challenge (air or oxygen) and independent component analysis (ICA). Hypoxia staining with pimonidazole correlated strongly with \acs{dOE-MRI} in a murine tumour model (SCCVII) and only weakly in a colorectal xenograft model (HCT-116). \acs{dOE-MRI} showed that hind limb muscle tumours were considerably more oxygenated than those implanted in the dorsal subcutaneous region. Finally, we provide strong evidence that treatment with Avastin improves tumour oxygenation in subcutaneous tumours. With \acs{dOE-MRI}, the sensitivity and speed of existing techniques was greatly improved. Since our technique requires no injectable contrast agent, special sequences or hardware, we anticipate that this technique can be quickly translated into the clinic. 


% Consider placing version information if you circulate multiple drafts
%\vfill
%\begin{center}
%\begin{sf}
%\fbox{Revision: \today}
%\end{sf}
%\end{center}
